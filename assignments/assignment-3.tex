\documentclass[12 pt]{article}

% Packages for mathematical symbols and equations
\usepackage[numbers]{natbib}
\usepackage{amsmath}
\usepackage{amssymb}
\usepackage{listings}
\usepackage{color}
\usepackage{setspace}
\usepackage{acro}
\usepackage{amsmath}
\usepackage{amsthm}
\usepackage{enumerate}
\usepackage{graphicx}
\usepackage{subcaption}
\usepackage{physics}
\usepackage{tikz}
\usetikzlibrary{quantikz2}
\usepackage{braket}
\usepackage[framemethod=tikz]{mdframed}
\usepackage{cancel}
\usepackage{empheq}
\usepackage{color}
\usepackage[english]{babel}
\usepackage{bbold}
\usepackage{amssymb}
\usepackage{titlesec}
\usepackage{adjustbox}
\usepackage{float}
%\numberwithin{equation}{section}
\usepackage{bm}
\usepackage{wrapfig}
\usepackage{relsize}
\usepackage[letterpaper ,top=1in,bottom=1in,left=1in,right=1in,marginparwidth=1.75cm]{geometry}
\usepackage[colorlinks=true, allcolors=blue]{hyperref}
\usetikzlibrary{calc,trees,positioning,arrows,chains,shapes.geometric,%
    decorations.pathreplacing,decorations.pathmorphing,shapes,%
    matrix,shapes.symbols}
\include{acronyms}
%\geometry{top=1.3in,bottom=1.3in}
\hypersetup{
    colorlinks,
    citecolor=black,
    filecolor=black,
    linkcolor=black,
  urlcolor=black}


\bibliographystyle{apsrev4-2}
% Set page margins

\title{Quantum Computation and Quantum Information\\
Assignment 3}
\author{Gabriel Balarezo}
\date{\today}

\begin{document}

\maketitle

\begin{mdframed}[backgroundcolor = gray!30,
  frametitle = Exercise 4.30]
  Suppose $U$ is a single-qubit unitary operation. Find a circuit containing $\mathcal{O}(n^2)$ Toffoli, CNOT and single qubit gates which implements a $C^n (U)$ gate (for $n>3$), using no work qubits. 
\end{mdframed}

So, the circuit we want to implement is the following:
\begin{equation*}
  \begin{quantikz}
    \lstick{$\ket{x_1}$} &  & \ctrl{1} &\\ 
    \lstick{$\ket{x_2}$} &  & \ctrl{1} & \\
    \lstick{$\ket{x_3}$} &  & \ctrl{2} & \\
    \vdots \\
    \lstick{$\ket{x_n}$} &  & \ctrl{1} & \\
    \lstick{$\ket{\psi}$} &  & \gate{U} &
    \end{quantikz}
\end{equation*}

For this, we will Lemma 6.1 from Ref. \cite{Barenco_1995}, which states that for any unitary $2\times 2$ matrix $U$, a $C^2(U)$ gate can be simulated by the following 
circuit:

\begin{equation*}
  \begin{quantikz}
    & \ctrl{1} & \midstick[3, brackets = none]{=} & & \ctrl{1} & & \ctrl{1} & \ctrl{2} &\\ 
    & \ctrl{1} & & \ctrl{1} & \targ{} & \ctrl{1} & \targ{} & &  \\
    & \gate{U} & & \gate{U^{1/2}} &         & \gate{(U^{1/2})^\dagger} &  & \gate{U^{1/2}} &
    \end{quantikz}
\end{equation*}

Proof of this lemma can be found in the same reference.\\

A generalisation of Lemma 6.1 can be found in Lemma 7.5 of the same reference, which states that for any unitary $2\times 2$ matrix $U$, a $C^{n-1}(U)$ gate can be simulated by the following circuit (illustrated for $n = 9$)

\begin{equation*} 
  \begin{quantikz}
    & \ctrl{1} & \midstick[10, brackets = none]{=} & & \ctrl{1} & & \ctrl{1} & \ctrl{1} &\\
    & \ctrl{1} &                                  & & \ctrl{1} & & \ctrl{1} & \ctrl{1} &\\ 
    & \ctrl{1} &                                  & & \ctrl{1} & & \ctrl{1} & \ctrl{1} &\\ 
    & \ctrl{1} &                                  & & \ctrl{1} & & \ctrl{1} & \ctrl{1} &\\ 
    & \ctrl{1} &                                  & & \ctrl{1} & & \ctrl{1} & \ctrl{1} &\\
    & \ctrl{1} &                                  & & \ctrl{1} & & \ctrl{1} & \ctrl{1} &\\ 
    & \ctrl{1} &                                  & & \ctrl{1} & & \ctrl{1} & \ctrl{1} &\\  
    & \ctrl{1} &                                  & & \ctrl{1} & & \ctrl{1} & \ctrl{2} &\\ 
    & \ctrl{1} &                                  &\ctrl{1}&\targ{}&\ctrl{1}&\targ{}& &\\ 
    & \gate{U} &                                  &\gate{V}&       &\gate{V^\dagger}&       & \gate{V} &
  \end{quantikz}
\end{equation*}

given $V$ a unitary matrix such that $V^2 = U$. Another version of this implementation can be found in Ref. \cite{liu2007analytic}

What we want to show is that the circuit above can be implemented using $\mathcal{O}(n^2)$ basic operations.   

We can notice that this circuit is a recursive implementation of Lemma 7.5. Let $C_{n-1}$ denote the cost of implementing a $C^{n-1}(U)$. We can also notice that the cost of applying $C^{1}(V)$ and $C^{1}(V^\dagger)$ gates is $\mathcal{O}(1)$. Corollary 7.4 from Ref.\cite{Barenco_1995} states that for an n-bit circuit (for $n\geq 7$), a $C^{n-2}(X)$ gate can be simulated using $\mathcal{O}(n)$ basic operations. The cost of implementing $C^{n-2}(V)$ gate is $C_{n-2}$ (by recursion). 
This implies that the cost of implementing the circuit above is $C_{n-1} = C_{n-2} + \mathcal{O}(n)$. This is a linear recurrence relation, and its solution is $C_{n-1} = \mathcal{O}(n^2)$.\\



Note: Illustrations were made using the package \texttt{quantikz2}. 


\bibliography{/home/pacman/Documents/git-repositories/quantum-computing/lecture-notes/references-2.bib}

\end{document}

