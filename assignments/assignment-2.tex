\documentclass[12 pt]{article}

% Packages for mathematical symbols and equations
\usepackage{amsmath}
\usepackage{amssymb}
\usepackage{listings}
\usepackage{color}
\usepackage{setspace}
\usepackage{acro}
\usepackage{amsmath}
\usepackage{amsthm}
\usepackage{enumerate}
\usepackage{graphicx}
\usepackage{subcaption}
\usepackage{../../../tex-packages/physics}
\usepackage{../../../tex-packages/qcircuit}
\usepackage[framemethod=tikz]{mdframed}
\usepackage{cancel}
\usepackage{empheq}
\usepackage{tikz}
\usepackage{color}
\usepackage[english]{babel}
\usepackage{bbold}
\usepackage{amssymb}
\usepackage{titlesec}
\usepackage{adjustbox}
\usepackage{float}
%\numberwithin{equation}{section}
\usepackage{bm}
\usepackage{wrapfig}
\usepackage{relsize}
\usepackage[letterpaper ,top=1in,bottom=1in,left=1in,right=1in,marginparwidth=1.75cm]{geometry}
\usepackage[colorlinks=true, allcolors=blue]{hyperref}
\usetikzlibrary{calc,trees,positioning,arrows,chains,shapes.geometric,%
    decorations.pathreplacing,decorations.pathmorphing,shapes,%
    matrix,shapes.symbols}
\include{acronyms}
%\geometry{top=1.3in,bottom=1.3in}
\hypersetup{
    colorlinks,
    citecolor=black,
    filecolor=black,
    linkcolor=black,
  urlcolor=black}



% Set page margins

\title{Quantum Computation and Quantum Information\\
Assignment 2}
\author{Gabriel Balarezo}
\date{\today}

\begin{document}

\maketitle

\begin{mdframed}[backgroundcolor = gray!30,
  frametitle = Exercise 1.2]
  Explain how a device which, upon input of one of two non-orthogonal quantum states $\ket{\psi}$ and $\ket{\varphi}$ correctly identified the state, could be used to build a device which cloned the states $\ket{\psi}$ and $\ket{\varphi}$, in violation of the non-cloning theorem. Conversely, explain how a device for cloning could be used to distinguish non-orthogonal quantum states.
\end{mdframed}

Given a device that can distinguish between non-orthogonal states $\ket{\psi}$ and $\ket{\varphi}$ (without measurement), we can then design a quantum circuit
that clones the states $\ket{\psi}$ and $\ket{\varphi}$ as we like.\\ 
First, let's define the non-orthogonal states $\ket{\psi}$ and $\ket{\varphi}$ as follows:
\begin{align}
  \ket{\psi} & \in \{\ket{\psi_+}, \ket{\psi_-}\} \\
  \ket{\varphi} & \in \{\ket{\varphi_+}, \ket{\varphi_-}\}
\end{align}
then, there exists a single-qubit operator, i.e. a rotation such that:

\begin{align*}
  U(\phi, \theta) \ket{\psi_+} = \ket{0} \\
  U(\phi, \theta)\ket{\psi_-} = \ket{1} \\
  U(\phi, \theta)\ket{\varphi_+} = \ket{0} \\
  U(\varphi, \theta)\ket{\varphi_-} = \ket{1}
\end{align*}
and 
\begin{align*}
  U^{-1}(\phi, \theta) \ket{0} = \ket{\psi_+} \\
  U^{-1}(\phi, \theta)\ket{1} = \ket{\psi_-} \\
  U^{-1}(\phi, \theta)\ket{0} = \ket{\varphi_+} \\
  U^{-1}(\varphi, \theta)\ket{1} = \ket{\varphi_-}
\end{align*}
Now, we can define a rotation from

\[\ket{0} \quad \text{to} \quad \cos\left(\frac{\phi}{2} \right) \ket{0} + \sin\left(\frac{\phi}{2} \right) e^{i \theta} \ket{1}  = \ket{\psi_+}\]
and from
\[\ket{1} \quad \text{to} \quad \sin\left(\frac{\phi}{2} \right) \ket{0} + \cos\left(\frac{\phi}{2} \right) e^{i \theta} \ket{1}  = \ket{\psi_-}\]
as follows:
\begin{equation}
  U(\phi, \theta) = 
  \begin{pmatrix}
  e^{i\theta/2} & 0 \\
  0 & e^{-i\theta/2}
  \end{pmatrix} 
  \begin{pmatrix}
  \cos\left(\frac{\phi}{2} \right) & \sin\left(\frac{\phi}{2} \right) \\
  \sin\left(\frac{\phi}{2} \right) & -\cos\left(\frac{\phi}{2} \right)
  \end{pmatrix}
\end{equation}
and the inverse transformation is given by
\begin{equation}
  U^{-1}(\phi, \theta) = 
   \begin{pmatrix}
  e^{-i\theta/2} & 0 \\
  0 & e^{i\theta/2}
  \end{pmatrix}
  \begin{pmatrix}
  \cos\left(\frac{\phi}{2} \right) & \sin\left(\frac{\phi}{2} \right) \\
  \sin\left(\frac{\phi}{2} \right) & -\cos\left(\frac{\phi}{2} \right)
  \end{pmatrix} 
\end{equation}
Now, we can define a quantum circuit that clones the states $\ket{\psi}$ and $\ket{\varphi}$ as follows:
\begin{align*}
  \Qcircuit @C=1em @R=1em {
    & \lstick{\ket{\psi_{\pm}}} & \gate{U(\phi, \theta)} & \ctrl{1}
    & \gate{U^{-1}(\phi, \theta)} & \qw & \rstick{\ket{\psi_{\pm}}} \\
    & \lstick{\ket{0}} & \qw & \targ
    & \gate{U^{-1}(\phi, \theta)} & \qw & \rstick{\ket{\psi_{\pm}}}
  }
\end{align*}
The same follows for the state $\ket{\varphi}$.\\

So far we have shown that a device that can distinguish between non-orthogonal states $\ket{\psi}$ and $\ket{\varphi}$ can be used to build a device that clones the states $\ket{\psi}$ and $\ket{\varphi}$. But for this we need to know if our qubit is in either 
$\ket{\psi}$ or $\ket{\varphi}$, so we can plug the correct phase and angle into the rotation operator, which is 
in contradiction with the non-cloning theorem.\\

Conversely, given the aforementioned cloning circuit, we can use it to distinguish between non-orthogonal states $\ket{\psi}$ and $\ket{\varphi}$ by

\begin{enumerate}
  \item Making n copies of our unknown state $\ket{\chi}$ which is either $\ket{\psi}$ or $\ket{\varphi}$. 
  \item Measuring the $\ket{\chi}$ state in the computational basis $\ket{\psi_{\pm}}$, i.e $\bra{\psi}\ket{\chi}$.
\end{enumerate}
if $\ket{\chi} = \ket{\psi}$, then the outcome is 1, and if $\ket{\chi} = \ket{\varphi}$, then the outcome is $\bra{\psi}\ket{\varphi}$.

\[
\text{Probability} \quad | \bra{\psi}\ket{\psi} |^2 = 1 \quad \text{and probability} \quad | \bra{\psi}\ket{\varphi} |^2 < 1
\]
Let $|\bra{\chi}\ket{\varphi}|^2 = p$, then the probability of measuring an overlap for n measurements of $\bra{\psi}\ket{\phi}$ is  
\[
p^n  = |\bra{\chi}\ket{\varphi}|^{2n}
\]
for example, $\displaystyle{\left \{ \ket{0}, \frac{\ket{0} + \ket{1}}{\sqrt{2}} \right\}}$
\[
p = \left| \bra{0} \left( \frac{\ket{0} + \ket{1}}{\sqrt{2}} \right) \right|^2 = \frac{1}{2}
\]
\[
p^n = \left( \frac{1}{2} \right)^n, \quad \text{and for four measurements} \quad p^4 = \left( \frac{1}{2} \right)^4 = \frac{1}{16}
\]
So we can see that as long as we can make n copies of the state $\ket{\chi}$, we can distinguish between non-orthogonal states $\ket{\psi}$ and $\ket{\varphi}$.\\

Even if $\ket{\psi}\sim \ket{\varphi}$, we can still distinguish between them by making enough n copies of the state $\ket{\chi}$ and measuring them in the appropriate 
basis.\\






























































\end{document}

